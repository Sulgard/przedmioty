\documentclass[12pt]{article}
%\usepackage[MeX]{polski} 
\usepackage[utf8]{inputenc}
\usepackage{graphicx}
\usepackage{amsmath} %pakiet matematyczny
\usepackage{amssymb} %pakiet dodatkowych symboli
\begin{document}

\begin{displaymath}
a + b \pm 4
\end{displaymath}

\begin{displaymath}
x \leqslant y %%żeby osiągnąć taki efekt trzeba zakomentować pakiet polski =)
\end{displaymath}

\begin{displaymath}
x \leq y
\end{displaymath}

\begin{displaymath}
A \subset B, C \subseteq D, E \backslash W, W`, R \cup T, F \cap K
\end{displaymath}

\begin{displaymath}
b \in P
\end{displaymath}

\begin{displaymath}
\alpha \beta \gamma \Gamma \pi \Pi \phi \varphi \mu \Phi
\end{displaymath}

\begin{displaymath}
\cos(20) = \cos^20 - \sin^2 0
\end{displaymath}

\begin{displaymath}
\tan(\pi)
\end{displaymath}

\begin{displaymath}
k_{n+1} = n^2 + k^{3n+1}_{n} - k_{n-2}
\end{displaymath}

\begin{displaymath}
f(n) = n^4 + 4n^2 - 2|_{n=12}
\end{displaymath}

\begin{displaymath}
x = a_0 + \frac{1}{a_1 + \frac{1}{a_2 + \frac{1}{a_3 + \frac{1}{a_4}}}}
\end{displaymath}

\begin{displaymath}
\sqrt{\frac{a}{b} + 3}
\end{displaymath}

\begin{displaymath}
\sqrt[n]{1 + x + x^2 + x^3 + ... + x^n}
\end{displaymath}

\begin{displaymath}
\sum^{10}_{i=1}t_i
\end{displaymath}


$\sum_{i=1}^{10}t_i$












\end{document}